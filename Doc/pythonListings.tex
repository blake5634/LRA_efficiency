\clearpage
\newpage
\begin{listing}[h]
    \caption{Python functions supporting the simulation and energy study.}\label{Listing1}
\end{listing}
\begin{minted}[linenos, fontsize=\small, frame=lines]{python}
import numpy as np
import seaborn as sns
import matplotlib.pyplot as plt
import sys
import control as ctl
import LRAsim as LRA
import heatmap as hm


TRAPEZOIDAL = LRA.TRAPEZOIDAL  # numerical integration method for energy


# get command line args:

if len(sys.argv) == 2:
    if sys.argv[1]=='plot':
        fname = input('enter heatmap filename: ')
        hm.hmap(fname.strip(),legend='Skin Dissipation ')
        quit()
    else:
        print('unknown argument: ', sys.argv[1])
        quit()

# get the energy flows

def RepHeatmap(wn,z, fd, heat):
    print(f'{wn:10.2f}, {z:10.5f}, {heat:.3e}', file=fd)

def Report2(wn,z, fd, eso, eca, elm, eld, ebs, Etot,yp):
    leakage = eso-(eld+ebs)
    plk = 100*leakage/eso
    print(f'{wn:10.2f}, {z:10.5f}, {eso:.3e}, {leakage:.3e}, {plk:8.1f}', file=fd)

nwn = 5
nzet = 5
fres = 150 # hz
wres = 2*np.pi*fres

wnv = np.geomspace(0.95*wres, 1.05*wres, num=nwn, endpoint=True)
zetav = np.geomspace(0.01, 1.0, num=nzet, endpoint=True)

dt = LRA.dt

fhz = fres
per = 1/fhz
Tmax = 400*per  # let's model 6 cycles

T = np.arange(0,Tmax,dt)
npts = len(T)

ncontact = 4  # four fingers touching (x4 skin model)
Ain = 1  # input amplitude (N)

sd={}
# sd['studytype'] = 'leakage'
# sd['legend']    = 'Energy Leakage (%)'
# sd['studytype'] = 'eout'
# sd['legend']    = 'Actuator Energy Output'
sd['studytype'] = 'SkinE'
sd['legend']    = 'Energy to Skin'

fname = f'heatmap{sd['studytype']}.csv'
dataf = open(fname, 'w')

#
#   input force to LRA
#
# Sin wave
# U = Ain*np.sin(wres*T)

# short rectangular pulses
U = np.zeros(len(T))
pf = 1/fres
p  = int(pf/dt)

for i,u in enumerate(U):
    if i%p==0:
        U[i] = Ain
    if (i+1)%p==0:
        U[i] = Ain
    if (i+2)%p==0:
        U[i] = Ain

dpar = {}
#  for all the wn and zeta possibilities:
#
for wn in wnv:
    for zetaLRA in zetav:
        #
        #   Simulate and analyze energy
        #

        # LRA properties
        M1 = 0.005 # kg   ()
        dpar['M1'] = M1
        # B1 = 0.03222  # Nsec/m
        K1 = wn**2 * M1
        dpar['K1'] = K1

        B1 = zetaLRA*2*np.sqrt(K1*M1)
        dpar['B1'] = B1

        print(f' Derived K1 ratio: {K1/2800:5f} vs 2800')
        # K1 = 2800  # N/m # (ref Jack's paper)

        # Case Properties
        Mc = 0.2250   # 225gr
        M2 = Mc
        dpar['M2']=M2

        # Skin Properties
        Kskin = 300 # N/m   (600-1200)
        K2=ncontact*Kskin     #
        dpar['K2'] = K2
        Bsk = (0.75+2.38)/2  # Nsec/m (0.75-2.38)
        # Bl = (2.38)  # Nsec/m (0.75-2.38)
        B3 = ncontact*Bsk
        dpar['B3']=B3

        Msk = 0.01 * M1  # essentially zero
        M3 = ncontact*Msk
        dpar['M3']= M3

        #
        # System Matrix
        a = -K1/M1
        b = -B1/M1
        c = K1/M1
        d = B1/M1

        e = K1/M2
        f = B1/M2
        g = (-K1-K2)/M2
        h = -B1/M2
        i = K2/M2

        j = K2/M3
        k = -K2/M3
        l = -B3/M3

        A = np.array([
            [0,1,0,0,0,0],
            [a,b,c,d,0,0],
            [0,0,0,1,0,0],
            [e,f,g,h,i,0],
            [0,0,0,0,0,1],
            [0,0,j,0,k,l]]  )

        B = np.array([
            [0],
            [-1/M1],
            [0],
            [1/M2],
            [0],
            [0]
            ] )
        C = np.identity(6)  # vector of all states
        D = np.zeros((6,1))           # no input coupling

        sdim = np.shape(A)[0]
        print('System Dimension: ', sdim)

        sys = ctl.ss(A,B,C,D)

        tp, yp = ctl.forced_response(sys, T, U)

        eso, eca, elm, eld, ebs, Etot = LRA.EnergyFlows(yp,U,dpar)
        wn_norm = wn/wres
        leakage = eso-(eld+ebs)
        plk = 100*leakage/eso
        if sd['studytype'] == 'leakage':
            heat = plk
        if sd['studytype'] == 'eout':
            heat = eso
        if sd['studytype'] == 'SkinE':
            heat = ebs
        RepHeatmap(wn_norm, zetaLRA, dataf, heat)

print('output map:', fname)
dataf.close()

print(f'Studytype: {sd["studytype"]}   legend: {sd["legend"]}')

hm.hmap(fname, studytype=sd['studytype'], legend=sd['legend'])
\end{minted}
%     \caption{Python code for LRA (Figure \ref{3MassSchematic}) simulation and energy study.}
% \end{breakablecode}


\clearpage
\newpage
\begin{listing}[h]
    \caption{Python functions supporting the simulation and energy study.}\label{Listing2}
\end{listing}
\begin{minted}[linenos, fontsize=\small, frame=lines]{python}

import numpy as np
import matplotlib.pyplot as plt
import sys
import control as ctl

#
#   Command line option
#
# if len(sys.argv) == 1:
#     RESONANT = True
# if len(sys.argv) == 2:
#     arg = sys.argv[1].lower()
#     print(f'got arg: ',arg)
#     if 'resonant'.startswith(arg):
#         RESONANT=True
#         zetaLRA = 0.01
#     else:
#         zetaLRA = 1.0
#         RESONANT=False

#
#   Compute energy flows
#

dt = 1.0e-4
TRAPEZOIDAL = True

def RMS(x,tr):
    idx = int((len(x)-1)*(1.0-tr)) # tr: pct of data used
    y = x[idx:]
    return np.sqrt(np.mean(x**2))


def integrate(x,xm1):
    if TRAPEZOIDAL:
        v =  0.5*(x+xm1)
        return v
    else:
        return x

timerange = 0.25  # percent data for energy analysis (from end)


def EnergyFlows(yp,U,d):
    # for forced_response
    x1  =  yp[0][:]   # MKS to mm
    xd1 =  yp[1][:]   # MKS to mm
    x2  =  yp[2][:]
    xd2 =  yp[3][:]
    x3  =  yp[4][:]
    xd3 =  yp[5][:]

    # energy accumulators
    eso = 0.0  # source output
    eld = 0.0  # LRA dissipation
    eca = 0.0  # einput to case
    elm = 0.0  # einput to LRA mass
    ebs = 0.0  # skin damping dissipation
    esk1 = 0.0  # e output to skin

    # delayed energy values for trapezoidal integration
    deld = {'eso':0, 'eld':0,'eca':0,'elm':0,'ebs':0,'esk1':0}


    # print(f'\n   Energy integration: {intmeth}')
    print(f'    (time range: last {100*timerange}%)')

    start = int(len(x1)*(1-timerange))
    end = len(x1)-1
    idxs = range(start,end,1)


    for i in idxs:
        # energy from source (has two outputs)
        f = U[i]
        dx = xd2[i]*dt
        e1 = f*dx   # energy to M2
        f = -U[i]
        dx = xd1[i]*dt
        e2 = f*dx   # energy to M1
        # eso += f*dx
        eso += integrate(e1+e2, deld['eso'])
        deld['eso'] =e1+e2

        # energy in LRA damper
        dx21 = (xd2[i]-xd1[i])
        f  = dx21*d['B1']   # damping force
        dx = dx21*dt   # length change
        # eld += f*dx
        eld += integrate(f*dx, deld['eld'])
        deld['eld'] = f*dx

        # energy into LRA mass
        f = -U[i]
        dx = xd1[i] * dt
        # elm += f*dx
        elm += integrate(f*dx, deld['elm'])
        deld['elm'] =   f*dx

        # energy to case
        f = U[i]
        dx = xd2[i]*dt
        # eca += f*dx
        eca += integrate(f*dx, deld['eca'])
        deld['eca'] =    f*dx

        # dissipation in Bskin
        f = d['B3']*xd3[i]
        dx = xd3[i]*dt
        # ebs += f*dx
        ebs += integrate(f*dx, deld['ebs'])
        deld['ebs'] =    f*dx

        # energy to skin
        f = d['K2']*(x2[i]-x3[i])
        dx = xd2[i]*dt
        # esk1 += f*dxs
        esk1 += integrate(f*dx, deld['esk1'])
        deld['esk1'] =    f*dx


    # get energy in the state variables (masses, springs)

    Etot = d['K1']*(x2[-1]-x1[-1])**2 + d['K2']*(x2[-1]-x3[-1])**2 + d['M1']*xd1[-1]**2 + d['M2']*xd2[-1]**2 + d['M3']*xd3[-1]**2

    return (eso, eca, elm, eld, ebs, Etot)



def LeakReport(eso, eca, elm, eld, ebs, Etot,yp):
    # for forced_response
    x1  =  yp[0][:]   # MKS to mm
    x3  =  yp[4][:]
    #
    #    Reports
    #
    print(f'Oscillation Amplitudes (rms):  LRA: {RMS(1000*x1,timerange):.3e}mm Skin: {1000*RMS(x3,timerange):.3e}mm')

    print(f'     Source energy (eso): {eso:.3e}\nSource flows:')
    print(f'           to Case (eca): {eca:.3e}')
    print(f'       to LRA mass (elm): {elm:.3e}')
    print(f'                   total: {eca+elm:.3e}\n')

    print(f'Energy sinks:')
    print(f'   LRA dissipation (eld): {eld:.3e}')
    print(f'  skin dissipation (ebs): {ebs:.3e}')
    print(f'                   total: {eld+ebs:.3e}\n')

    print(f'kinetic+potential (Etot): {Etot:.3e}')


    leakage = eso-(eld+ebs)
    print(f'              difference: {leakage:.3e}')
    print(f'          Energy leakage: ({100*(leakage)/(eso):.1f}%)')
\end{minted}
